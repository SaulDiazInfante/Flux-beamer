\begin{frame}{Formulating a SIS-CTMC }
    \begin{textblock*}{60mm}(20mm, 25mm)
        Let 
        $
                \{ I_t\}_{t \geq 0}, p_i(t) = \probX{I_t = i}.
        $
        \only<2->{
            Thus, the Markov property becomes in
            \begin{align*}
                \probCX{ I_{t_{n + 1}}}{I_{t_0}, \cdots, I_{t_n} }
                    &=
                    \probCX{ I_{t_{n + 1}}}{I_{t_n}}
                    \\
                    \text{for all } &
                    t_0 < t_1 < \cdots < t_n 
            \end{align*}
        }
    \end{textblock*}
    \only<3>{
        \begin{textblock*}{90mm}(20mm, 50mm)
            \begin{align*}
                &p_{ji}(\Delta t):=
                    \begin{cases}
                        \frac{\beta i (N - i)}{N} \Delta t 
                            + o(\Delta t),     
                            & j = i + 1
                        \\
                        (b + \gamma) i \Delta t 
                            + o(\Delta t),
                            &   j = i - 1
                        \\
                        1 - \left [
                                \frac{\beta i (N - i)}{N} +
                                (b + \gamma) i %                
                            \right] \Delta t
                            + o(\Delta t) , 
                            & j=i
                        \\
                        o(\Delta t) & \text{otherwise}
                    \end{cases}
                \\
                & \lim_{t\to \infty}
                    \frac{o(\Delta t)}{\Delta t}
                    = 0
            \end{align*}
        \end{textblock*}
    }
\end{frame}
 %%%%%%%%%%%%%%%%%%%%%%%%%%%%%%%%%%%%%%%%%%%%%%%%%%%%%%%%%%%
\begin{frame}{}
    \begin{textblock*}{90mm}(10mm, 5mm)
        Using the notation for birth and death
        processes, we have
        \begin{equation*}
            p_{ij}(\Delta t):=
                \begin{cases}
                    b(i) \Delta t + o(\Delta t),     
                        & j = i + 1
                    \\
                    d(i) \Delta t + o(\Delta t), 
                        &   j = i - 1
                    \\
                    1 - \left [
                            b(i) + d(i) %                
                        \right] \Delta t
                        + o(\Delta t), 
                        & j=i
                    \\
                    0 & \text{otherwise}.
                \end{cases}
        \end{equation*}
    \end{textblock*}
    \only<2->{    
        \begin{textblock*}{90mm}(10mm, 30mm)       
            $\probX{I_0 = i_0} = 1$,
            \begin{align*}
                p_i(t + \Delta t)
                    =&
                        p_{i - 1} (t) b(i -1) \Delta t
                        \\
                        & + 
                            p_{i + 1} (t) d(i + 1) \Delta t
                        \\
                        & +
                            p_{i} (t)[1 - (b(i) + d(i))] \Delta t
                        + o(\Delta t)
                \\
                i &= 1, 2, \dots, N
            \end{align*}
        \end{textblock*}
    }
    \only<3->{
        \begin{textblock*}{90mm}(10mm, 60mm)              
            Thus
            \begin{align*}
                \frac{p_i (t - \Delta t) - p_i (t) }{\Delta t}
                = & 
                p_{i - 1} (t) b(i-1)
                    +
                    p_{i + 1} (t) d(i+1)  
                \\
                    &-
                    p_i [b(i) + d (i)] 
                    +o(\Delta t)
                \\
                & i = 1,2,\cdots, N.
            \end{align*}
        \end{textblock*}
    }
    \end{frame}
%%%%%%%%%%%%%%%%%%%%%%%%%%%%%%%%%%%%%%%%%%%%%%%%%%%%%%%%%%%%%%%%
\begin{frame}{}
    \begin{textblock*}{60mm}(10mm, 10mm)       
        Hence, letting $\Delta t \to 0$,  we obtain
        \begin{align*}
            \frac{dp_i(t)}{dt}
            = & 
            p_{i - 1} (t) b(i-1)
                +
                p_{i + 1} (t) d(i+1)  
            \\
                &-
                p_i [b(i) + d (i)] 
            \\
                & i = 1,2,\cdots, N.
        \end{align*}
    \end{textblock*}
    \only<2->{        
        \begin{textblock*}{60mm}(65mm, 10mm)       
            \begin{empheq}[box=\fbox]{align*}
                FKE:  &\frac{dp}{dt} = Q p
                \\
                p(t) & = (p_0(t), \dots, p_N(t)) ^ {\top}
            \end{empheq}
        \end{textblock*}
    }
    \only<3->{
        \begin{textblock*}{90mm}(10mm, 30mm)
            \begin{equation*}
                Q = 
                \begin{pmatrix}
                    0   & d(1)  & 0 & \dots     & 0
                    \\
                    0   & -[b(1) + d(1)]    & d(2)  & \dots     & 0
                    \\
                    0   & b(1)  & -[b(2) + d(2)]    & \dots     & 0     
                    \\
                    0   & 0     & b(2)  & \dots     & 0
                    \\
                    \vdots  & \vdots    & \vdots    & \ddots    & \vdots
                    \\
                    0   & 0     & 0     & \dots     & d(N)
                    \\
                    0   & 0     & 0     & \dots     & -d(N)    
                \end{pmatrix}
            \end{equation*}
            \only<4->{
                Results that
                $$
                    \lim_{t \to \infty}
                        p(t) = (1, 0, \dots, 0)^{\top}
                $$
                and
                $$
                    Q = \lim_{\Delta t \to 0}
                        \frac{P(\Delta t) - I}{\Delta t}
                $$
            }
        \end{textblock*}
    }
\end{frame}
%%%%%%%%%%%%%%%%%%%%%%%%%%%%%%%%%%%%%%%%%%%%%%%%%%%%%%%%%%%%%%%%%%%%%%%%%%%
\begin{frame}{Expected value of the SIS-CTMC}
    \begin{textblock*}{50mm}(10mm, 20mm)
        Consider the m.g.f    
        \begin{align*}
            M(\theta, t)&:= 
            \E [\exp( \theta I_t)]
            \\
            &= \sum_{i = 0} ^ N
                p_i(t) \exp( i \theta)
        \end{align*}
    \end{textblock*}   
    \only<2->{
        \begin{textblock*}{60mm}(65mm, 20mm)
            Results that
            $$
                \E[I_t ^ k] =
                    \frac{\partial ^ k M}{\partial \theta ^ k}
                    \big |_{\theta = 0},
                    \quad k=1,2 , \dots, 
            $$
            
            
        \end{textblock*}
    }
    \only<3>{
        \begin{textblock*}{120mm}(15mm, 50mm)
            Now we deduce a differential equation
            for the moments of our sis-CTMC.
        \end{textblock*}
    }
\end{frame}
%%%%%%%%%%%%%%%%%%%%%%%%%%%%%%%%%%%%%%%%%%%%%%%%%%%%%%%%%%%%%%%%%%%%%%%%%%%
\begin{frame}{}
    \begin{textblock*}{100mm}(10mm, 10mm)
        \begin{align*}
            \frac{\partial M}{\partial t}
                =&
                    \sum_{i = 0} ^ N 
                        \frac{d p_i}{dt}
                        \exp(i \theta)
                \\
                \only<2>{
                    \text{from r.h.s of  FKE} &
                    \\
                        = &
                        \exp(\theta) 
                            \sum_{i = 0} ^N 
                                p_{i - 1} \exp[(i - 1) \theta] b(i-1)
                    \\
                        & +
                        \exp(- \theta)
                            \sum_{i = 0} ^N 
                                p_{i + 1} \exp[(i + 1) \theta] d(i + 1)
                    \\
                        & -
                        \sum_{i = 0} ^ N
                            p_i \exp(i \theta) (b(i) + d(i))
                }
        \end{align*}
    \end{textblock*}
\end{frame}
%%%%%%%%%%%%%%%%%%%%%%%%%%%%%%%%%%%%%%%%%%%%%%%%%%%%%%%%%%%%%%%%%%%%%%%%%%%
\begin{frame}{}
    Substituting definition of $b, d$
    we obtain
    \begin{align*}
        \frac{\partial M}{\partial t} 
        =&
            \beta(exp(\theta) -1) 
                \sum_{i = 1}^N i 
                    p_i \exp(i \theta)
        \\    
        &+ 
            (b + \gamma)(\exp(- \theta) -1) 
                \sum_{i=1}^N 
                    i p_i \exp(i\theta)
        \\
        & -
            \frac{\beta}{N} (\exp(\theta) - 1)
                \sum_{i=1}^N
                    i ^ 2 p_i \exp(i \theta)
        \\
        \only<2>{
            = &
                [
                    \beta(exp(\theta) -1)
                    (b + \gamma)(\exp(- \theta) -1)
                ] \frac{\partial M}{ \partial \theta}
            \\
            & -
                \frac{\beta}{N} (\exp(\theta) - 1)
                \frac{\partial ^ 2 M}{\partial \theta^2}
        }
    \end{align*}
\end{frame}
%%%%%%%%%%%%%%%%%%%%%%%%%%%%%%%%%%%%%%%%%%%%%%%%%%%%%%%%%%%%
\begin{frame}{}
    \begin{bibunit}[apalike]
        Following \cite{Bailey1964} we can deduce from the above equation
        \begin{equation*}
        \frac{d \E (I_t)}{dt} =
            [\beta - (b + \gamma)] \E(I_t)
            - \frac{\beta}{N}
                \E(I_t ^2).
        \end{equation*}
        Then we conclude as in the SIS-DTMC.
        \putbib
    \end{bibunit}
\end{frame}
%%%%%%%%%%%%%%%%%%%%%%%%%%%%%%%%%%%%%%%%%%%%%%%%%%%%%%%%%%%%%  
\begin{frame}{}
    Using the Guillespie algorithm
         \includegraphics[width=\textwidth, keepaspectratio]%
         {./assets/random_walk_SIS-CTMC.png}
\end{frame}
