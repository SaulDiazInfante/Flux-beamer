 \begin{frame}{}% clea[label=ExistenciaMao,noframenumbering]%
    %{Apendice A: Existencia y unicidad (condiciones locales)}        
%    \begin{overlayarea}{\textwidth}{1.0\textheight} 
%     \begin{columns}
%         \column{.57\textwidth}
%             \begin{theorem}
%                  (EU-1)-(EU-3)              
%                 $\Rightarrow$ 
%                 $\exists ! \  \{y(t)\}_{t\geq 0}$, $\forall y(0)=y_0\in 
%                 \mathbb{R}^d$. 
%                 \\              
%                 Adem\'as $0<T<\infty$,
%                 \begin{itemize}[<+-|alert@+>]
%                     \item               
%                     $
%                         \EX{y(T)}< 
%                             \left(
%                                 |y_0|^2 +2\alpha T 
%                             \right)\exp(2\beta T),
%                     $
%                     \item               
%                     $\tau_n := \inf \{ t\geq 0 : |y(t)|>n\}$, $n\in \N$,         
%            
%                     \only<3->{
%                         $$  
%                             \textcolor<3>{red}{
%                                 \prob{[\tau_n\leq T]}
%                                 \leq \frac{
%                                 \left(
%                                 |y_0|^2 +2\alpha T 
%                                 \right)
%                                 \exp(2\beta T)
%                                 }{n},
%                             }
%                         $$ 
%                     }   
%                     \item<4>                    
%                         $
%                             \textcolor<4>{red}{
%                                 \EX{|y(t)|^p}
%                                 \leq
%                                 2^{\frac{p-2}{2}}
%                                 \left(
%                                 1 + \EX{|y_0|^p}
%                                 \right)e^{Cpt}.             
%                             }
%                         $           
%                 \end{itemize}           
%             \end{theorem}
%         \column{.5\textwidth}       
%             \begin{bibunit}[apalike]        
%                 \cite{Mao2013}          
%                 \biblio{PhdThesisBib.bib}
%             \end{bibunit}
%     \end{columns}
%     \hyperlink{Construccion<1>}{
%         \beamerreturnbutton{Construccion}
%     }
%    \end{overlayarea}
 \end{frame}
% 
%%%%%%%%%%%%%%%%%%%%%%%%%%%%%%%%%%%%%%%%%%%%%%%%%%%%%%%%%%%%%%%%%%%%%%%%%%%%%%%%
% \begin{frame}[label=MatrixFunctions,noframenumbering]
%     \frametitle{Apendice A}
%     \scalebox{0.85}{\parbox{1.0\linewidth}{
%         \begin{align*}  
%             A^{(1)}(h,u)&:=
%             \begin{pmatrix}
%                 e^{ha_1(u)} & 
% \multicolumn{2}{c}{\text{\kern0.5em\smash{\raisebox{-1ex}{\huge 0}}}} \\
%                 &\ddots\\
%                 
% \multicolumn{2}{c}{\text{\kern-0.5em\smash{\raisebox{0.95ex}{\huge 0}}}} 
%                 & e^{ha_d(u)}
%             \end{pmatrix},
%             \\
%         %   
%             A^{(2)}(h,u)&:=
%             \begin{pmatrix}
%                 \left(
%                     \displaystyle
%                     \frac{e^{ha_1(u)} - 1}{a_1(u)}
%                 \right)\1{E_1^c}    & 
%                 \multicolumn{2}{c}{\text{\kern0.5em\smash{\raisebox{-1ex}{\huge 
% 0}}}}\\
%                 & \ddots&\\
%                 \multicolumn{2}{c}{\text{\kern0.5em\smash{\raisebox{-1ex}{\huge 
% 0}}}}&
%                 \left(
%                     \displaystyle
%                     \frac{e^{ha_d(u)} - 1}{a_d(u)}
%                 \right)\1{E_d^c}% + h \1{E_i} 
%             \end{pmatrix}
%             +h
%             \begin{pmatrix}
%                 \1{E_1} & 
% \multicolumn{2}{c}{\text{\kern0.5em\smash{\raisebox{-1ex}{\huge 0}}}}\\
%                 &\ddots &\\
%                 \multicolumn{2}{c}{\text{\kern0.5em\smash{\raisebox{-1ex}{\huge 
% 0}}}} &
%                 \1{E_d}
%             \end{pmatrix},\\    
%             E_j&:=\{x \in \R^d: a_j(x)=0\} , \qquad 
%             b(u):= 
%             \left(
%                 b_1(u^{(-1)}), \dots , b_d(u^{(-d)})
%             \right)^T.      
%         \end{align*}
%         }
%     }
%     \\
%     \hyperlink{Construccion<6>}{\beamerreturnbutton{Teorema}}
% \end{frame}
% %
% \begin{frame}[noframenumbering]
%     \frametitle{Apendice B: Resultado para ceros aislados}    
%     \begin{columns}
%         \column{.4\textwidth}
%         \begin{definicion}[DD respecto a $p$]
%              $u,\mathbf{p}\in \R^2$,  $\alpha$ angulo positivo respecto a 
%             eje-$x$ 
%             segmento $\overline{u \mathbf{p}}$.  
%             \begin{align*}
%                 f_{\alpha}(u) &= 
%                 \frac{ \innerprod{q-u}{\nabla f(u)}}{|u-q|}         
%             \end{align*}
%              \emph{derivada direccional respecto $\mathbf{p}$ en $u$}.
%         \end{definicion}
%         \begin{definicion}[Star-like set]
%             $S\subset \R^2$ es \emph{star-like} respecto $\mathbf{p}$,  $\forall 
%                 \ s \in S$ 
%             el segmento abierto $\overline{s \mathbf{p}}$  esta en $S$.
%         \end{definicion}
% 
%         \column{.6\textwidth}
%         %\includegraphics[width=\linewidth]{Imagenes/Apendice/LawlorThm.png}
%         \begin{theorem}
%             \begin{itemize}
%                 \item 
%                     $\mathbf{p}\in \R^2$, $S\subset \R^2$ star-like respecto 
%                     $\mathbf{p}$ en el dominio de $f$,$g$.
%                 \item
%                     En $S$, $f,g$ diferenciables , $g_{\alpha}(s)\neq 0$,   
%                 \item 
%                     $f(\mathbf{p})=g(\mathbf{p})=0$,
%                     \quad
%                     $
%                         \displaystyle
%                         \lim_{x \to \mathbf{p}}
%                         \frac{f_{\alpha}(x)}{g_{\alpha}(x)} = L,    
%                     $
%             \end{itemize}
%             Entonces
%                 $ 
%                     \displaystyle
%                     \lim_{x \to \mathbf{p}}
%                     \frac{f(x)}{g(x)} = L.
%                 $
%         \end{theorem}
%         \begin{bibunit}[alpha]
%             \nocite{FineAIandKass1966}
%             \biblio{PhdThesisBib.bib}
%         \end{bibunit}
%     \end{columns}
% \end{frame}
% % 
% %%%%%%%%%%%%%%%%%%%%%%%%%%%%%%%%%%%%%%%%%%%%%%%%%%%%%%%%%%%%%%%%%%%%%%%%%%%%%%%%
% % \begin{frame}[noframenumbering]
% \end{frame}
