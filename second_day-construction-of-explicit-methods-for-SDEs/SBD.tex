%%%%%%%%%%%%%%%%%%%%%%%%%%%%%%%%%%%%%%%%%%%%%%%%%%%%%%%%%%%%%%%%%%%%%%%%%%%%%%%%%%%%%%%%%%%%%%%%
\begin{frame}
	\begin{bibunit}[apalike]
		\frametitle{Nuestra idea}
		\hypertarget{Idea}{}    
		En 2005, Matus et. al., usan una versión del \hyperlink{dfn:Steklov}{\structure{promedio de Steklov}}, 
		para logra un esquema en diferencias \structure{exacto} que resolve EDOs no lineales de la forma
		\begin{equation*}
			\frac{dx}{dt}=f_1(x)f_2(t)
		\end{equation*}
		\nocite{matus2005exact}
		\biblio{BibliografiaTesis}
	\end{bibunit}
\end{frame}

%%%%%%%%%%%%%%%%%%%%%%%%%%%%%%%%%%%%%%%%%%%%%%%%%%%%%%%%%%%%%%%%%%%%%%%%%%%%%%%%%%%%%%%%%%%%%%%%%
\begin{frame}%[label=frm:12]
	\frametitle{Método Steklov para EDEs escalares}
	Queremos aproximar:
	\begin{align*}
		dy(t)&=f(t,y(t))dt + g(t,y(t))dW_t, \quad y_0=cte\quad t\in[0,T],\\
		f,g&:[0,T]\times \R \to \R.
	\end{align*}
	\begin{block}{Considerando su forma integral:}
		\begin{equation*}
			y(t) = y_0 + \int_{0}^{t} f(s,y(s))ds
			+\int_{0}^t g(s, y(s))dW_s
		\end{equation*}
	\end{block}
\end{frame}
% %%%%%%%%%%%%%%%%%%%%%%%%%%%%%%%%%%%%%%%%%%%%%%%%%%%%%%%%%%%%%%%%%%%%%%%%%%
\begin{bibunit}[apalike]
\begin{frame}%[label=frm:13]
	\frametitle{Existencia y unicidad de soluciones}
	\begin{overlayarea}{\textwidth}{.5\textwidth}
	\only<+>{
	\begin{block}{Sean $f,g:\R \to \R$.}
	  Hipótesis:
	  \begin{itemize}
	  	\item $f(t,x) = f_1(t)f_2(x)$.
		\item 
			\structure{Lipschitz globales}. $\exists \ L_1>0$ t.q. $\forall x,y \in \R$, $t\in [0,T]$
			\begin{align*}
				&|f(x,t)-f(y,t)|^2 \vee |g(x,t)-g(y,t)|^2 \leq L |x-y|^2.
			\end{align*}
		\item
			\structure{Crecimiento Lineal}. $\exists \ L_2>0$ t.q. $\forall x,y \in \R$, $t\in [0,T]$
			\begin{align*}
				 &|f(x,t)|^2 \vee |g(x,t)|^2 \leq L_2 (1+|x|^2).
			\end{align*} 
	  \end{itemize}
	\end{block}
	}
	\only<+->{
	  \begin{block}{Bajo estos supuestos  $\exists ! \ y(t)$ t.q.}
		$$
		\mathbb{E}
		\left(
			\int_{0}^T|y(t)|^2dt
		\right)<\infty
		$$
		\nocite{Mao2007}
	  \end{block}
	}
  \end{overlayarea}
  \only<2>{\biblio{PhdThesisBib.bib}}
  \end{frame}
\end{bibunit}
%%%%%%%%%%%%%%%%%%%%%%%%%%%%%%%%%%%%%%%%%%%%%%%%%%%%%%%%%%%%%%%%%%%%%%%%
\begin{frame}%[label=frm:14]{}
  \frametitle{Construcción de métodos Tipo Euler}
	  \begin{columns}
	  	\column{.3\textwidth}
				  Tipo base: \structure{Euler-Maruyama (EM)}.	  
		\column{.8\textwidth}
			\vspace*{-.6cm}
			\only<3>{
				\begin{empheq}[box=\shadowbox*]{equation*}
					y({t_{n+1}})=y_{t_{n}}+
					\int_{t_n}^{t_{n+1}}f(y(s))ds 
					+ \int_{t_n}^{t_{n+1}} g(y(s)) dW_s
					\tag{*}
				\end{empheq}
			}
	  \end{columns}
  %
  \begin{overlayarea}{\textwidth}{.5\textwidth}
	\only<1-2>{
	\begin{block}{Discretizamos $[0,T]$ con un  paso uniforme $h$:}
	  \begin{itemize}
		\item $t_n=nh$ $n=0,1,2,\dots, N$.
		\item $Y_n \approx y({t_n})$
	  \end{itemize}
	\end{block}
	}
	\only<2>{
	\begin{block}{Para cada nodo}
		\begin{align*}
		  y({t_{n+1}})&=y_{t_{n}}+
		  \underbrace{
		  	\int_{t_n}^{t_{n+1}}f(y(s))ds}_{\approx \text{Con algún método}
		  }
		  + \underbrace{
				\int_{t_n}^{t_{n+1}}
					g(y(s)) dW_s	
			}_{\approx  g(y_{t_n}) \Delta W_n}\\
			\Delta W_n&:= 
				\left(
						W_{t_{n+1}}-W_{t_{n}}
				\right) \sim \sqrt{h}\calN(0,1).
		\end{align*}
	  \end{block}
	}
	\only<3>{
	\begin{exampleblock}{Para el Euler-Mayurama se considera}
		$ \displaystyle
		 	\int_{t_n}^{t_{n+1}} f(y({s}))ds 
			 	\approx f(Y_n)h
		$,\\
		\vspace*{.25cm}
		EM  para (*) :
		$\displaystyle
			\textcolor{orange}{
				Y_{n+1}=Y_n+f(Y_n)h + g(Y_n) \Delta W_n,
			}
			\quad n=0,1\dots, N-1,\ Y_0 = y_0.		
		$
	\end{exampleblock}
	}
  \end{overlayarea}
\end{frame}
%%%%%%%%%%%%%%%%%%%%%%%%%%%%%%%%%%%%%%%%%%%%%%%%%%%%%%%%%%%%%%%%%%%%%%%%%%%%
\begin{frame}%[label=frm:15]{}
	\frametitle{Promedio especial de Steklov}
	\begin{overlayarea}{\textwidth}{.8\textwidth}
	\vspace*{1cm}
	 \begin{tcolorbox}[title =Estimamos la deriva con el promedio especial de Steklov ]
		 \only<1-2>{
			\begin{align*}
				\textcolor{cyan}{
					f(y(t))
				} 
				&\approx 
				\textcolor<1-3>{magenta}{
					\varphi(Y_n, Y_{n+1})
					:=
					\left(
						\frac{1}{Y_{n+1}-Y_{n}}  
						\int_{Y_n}^{Y_{n+1}} \frac{du}{f(u)}
					\right)^{-1},			
				}\\
				t_n\leq & t \leq t_{n+1},\\
				Y_n=&Y_{t_n}, \quad t_n=nh.
			\end{align*}
		}
		\only<3>{
			\begin{align*}
				\textcolor{cyan}{
					f(y(t))
				} 
				&\approx
				\varphi(Y_n, Y_{n+1})
				:=
				\underbrace{
					\left(
						\textcolor{red}{		
							\frac{1}{Y_{n+1}-Y_{n}}  
							\int_{Y_n}^{Y_{n+1}} \frac{du}{f(u)}
						}
					\right)^{-1}
				}_{\textcolor{orange}{Restrictivo?}}
			\end{align*}
		}
	\end{tcolorbox}
	\only<2>{
		\begin{block}{Aproximamos}
			\begin{equation*}
				\int_{t_n}^{t_{n+1}}f(y(s))ds 
				\textcolor{orange}{
					\approx \varphi(Y_n, Y_{n+1})h
				}
			\end{equation*}
		\end{block}
	}
	\begin{bibunit}[apalike]
		\nocite{matus2005exact}
		\only<1>{
			\biblio{BibliografiaTesis}
		}
	\end{bibunit}
	\end{overlayarea}   
\end{frame}

%%%%%%%%%%%%%%%%%%%%%%%%%%%%%%%%%%%%%%%%%%%%%%%%%%%%%%%%%%%%%%%%%%%%%%%%%%%%
\tikzstyle{na} = [baseline=-.5ex]
\tikzstyle{blockYellow}= [rectangle, draw, fill=yellow!40,
    text width=6em, text centered, rounded corners, minimum height=4em]

\tikzstyle{blockGreen}= [rectangle, draw, fill=green!40,
    text width=10em, text centered, rounded corners, minimum height=4em]

\tikzstyle{blockGray}= [rectangle, draw, fill=gray!40,
    text width=6em, text centered, rounded corners, minimum height=4em] 
\tikzstyle{every picture}+=[remember picture]
\everymath{\displaystyle}
%%%%%%%%%%%%%%%%%%%%%%%%%%%%%%%%%%%%%%%%%%%%%%%%%%%%%%%%%%%%%%%%%%%%%%%%%%%%%%
\begin{frame}
	\frametitle{Métodos Steklov }
	\begin{block}{Familia Steklov}
		\begin{equation*}
			Y_{n+1}=
				\tikz[baseline]{
					\only<1-3>{
						\node[fill=blue!20,anchor=base] (t1)
							{$Y_n+\varphi(Y_n, Y_{n+1}) h$};
						}
					\only<4->{
						\node[fill=green!20,anchor=base] (t1)
							{$Y_n^{\star}$};
					}
				}
					+
				\tikz[baseline]{
					\only<1-3>{
						\node[fill=red!20,anchor=base] (t2)
						{$g(Y_n)\Delta W_n$};
					}
					\only<4->{
						\node[fill=red!20,anchor=base] (t2)
							{$g(Y^{\star}_n)\Delta W_n$};
					}
				}
		\end{equation*}
	\end{block}
	\begin{overlayarea}{\textwidth}{.5\textheight}
		\begin{columns}
			\column{.7\textwidth}
				\begin{itemize}
					\item <2-> 
						\tikz[na] \node [blockYellow] (n1) 
							{$\approx\int_{Y_n}^{Y_{n+1}} \frac{du}{f(u)}$ (Cuadraturas)};
					\item<4-> 
						\tikz[na] \node[blockGreen] (n3) 
							{$Y_n^{\star} = Y_n + h \varphi(Y_n, Y_n^{\star})$ \\ (Split-Step)};
				\end{itemize}
			\column{.5\textwidth}
			\begin{itemize}
				\item <3-> 
					\tikz[na] \node[blockGray] (n2) 
						{$\varphi(Y_n, Y_{n+1}^*)$\\(Pre-Corr)};
			\end{itemize}
		\end{columns}	
		\begin{tikzpicture}[overlay]
			\path [->]<2->    (t1) edge[bend right]  (n1);%
			\path [->]<3->    (t1) edge[bend left]   (n2);%
			\path [->]<4->    (t1) edge[bend left]   (n3);%
		\end{tikzpicture}
	\end{overlayarea}
\end{frame}
%%%%%%%%%%%%%%%%%%%%%%%%%%%%%%%%%%%%%%%%%%%%%%%%%%%%%%%%%%%%%%%%%%%%%%%%%%%%%%%%%%%%%%%%%%%%%%%%%
\tikzset{
	state/.style={
		rectangle,
		rounded corners,
		draw=black, very thick,
		minimum height=1em,
		inner sep=1pt
		%text centered
	},
	invisible/.style={opacity=0},
	visible on/.style={alt={#1{}{invisible}}},
	alt/.code args={<#1>#2#3}{%
		\alt<#1>{\pgfkeysalso{#2}}{\pgfkeysalso{#3}} % \pgfkeysalso doesn't change the path
	},
}
\begin{frame}
	\frametitle{Steklov Explicito}
	\vspace*{.25cm}
	\begin{tikzpicture}[->,>=stealth']
		\node[
			state,
			text width=5.5cm,
			fill = gray!20
		] (SDE)
		{
			\vspace*{-0.5cm}
			\begin{align*}
				dy(t) &= {\color{red} f(t,y(t))dt} + g(t,y(t))dW_t \\
				f(t,y(t)) &= f_1(t) f_2(y(t))
			\end{align*} 
		}; 		
		\node[
			state,
			below of=SDE,
			yshift=-2cm,
			anchor=center,
			visible on=<2->
		] (SID) 
		{
			\begin{tabular}{l}
				\textbf{Steklov Implicito Determinista}\\
				\parbox{4.5cm}{
					$
						y_{n+1} = y_n + h \varphi_1(t_n)\varphi_2(y_n,y_{n+1})
					$	
				}\\[.5em]
				\textbf{Define}\\
				\parbox{4cm}{
					$
						H(x):= \int_{0}^{x}
							\frac{du}{f_2(u)}
					$	
				}
			\end{tabular}
		};
		% State: Steklov Equation
		\node[
			state,    	% layout (defined above)
			text width=5.5cm, 	% max text width
			yshift=2.25cm, 		% move 2cm in y
			right of=SID, 	% Position is to the right of QUERY
			node distance=6.0cm, 	% distance to QUERY
			anchor=center,
			visible on=<3->
		] (SE) 	% posistion relative to the center of the 'box'
		{%
			$\displaystyle
				y_{n+1} - y_n
				=
				\varphi_1(t_n)\frac{y_{n+1}-y_n}{H(y_{n+1})-H(y_{n})}h
			$
		};
		
		% State: Steklov Explicito Determinista
		\node[
			state,
			below of=SE,
			yshift=-3cm,
			anchor=center,
			text width=5.2cm,
			visible on=<5->
		] (SED)  
		{%
			\textbf{Steklov Explicito Determinista}
			\begin{align*}
				y_{n+1} &= \Psi_h(t_n, Y_n) \\
				\Psi_h(t_n, Y_n)&:= H^{-1}
				\left[
				H(y_n) + h \varphi_1(t_n) 
				\right]
			\end{align*}
		};
		
		% State: Steklov Explicito Stocastico
		\node[
			state,
			below of=SID,
			yshift=-1.5cm,
			anchor=center,
			text width=5cm,
			fill=green!40,
			visible on=<6->
		] (SES) 
		{%
			\textbf{Steklov Explicito Estoc\'astico}
			\begin{align*}
				Y_{n+1} &= \Psi_h(t_n, Y_n) + g(t_n, Y_n)\Delta W_n \\
			\end{align*}
		};
		
		% draw the paths and and print some Text below/above the graph
		\path [->]<2-> (SDE)  edge   (SID) ;
		\path [->]<3-> (SID)  edge[bend left=25]   (SE) ;
		\path [->]<4-> (SE)   edge    node[anchor=left,right]{Resolviendo $y_{n+1}$} (SED);
		\path [->]<6-> (SED)  edge [bend right=-20] (SES);
	\end{tikzpicture}
\end{frame}